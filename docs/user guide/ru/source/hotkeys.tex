\newpage
\section{Список горячих клавиш}

\begin{tabular}{|c|l|}
\hline
\textit{Сочетание клавиш} & \textit{Назначение} \\
\hline
\multicolumn{2}{|c|}{\textbf{Главное меню}} \\
\hline
Ctrl+N & Создание нового документа \\
\hline
Ctrl+O & Открытие существующего документа \\
\hline
Ctrl+S & Сохранить текущий документ \\
\hline
Ctrl+Shift+S & Сохранить все открытые на текущий момент документы \\
\hline
Ctrl+W & Закрыть текущую вкладку \\
\hline
Ctrl+Shift+W & Закрыть все открытые вкладки\\
\hline
\multicolumn{2}{|c|}{\textbf{Меню редактирования}} \\
\hline
Ctrl+X & Вырезать \\
\hline
Ctrl+C & Копировать\\
\hline
Ctrl+V & Вставить \\
\hline
Ctrl+Z & Отменить действие \\
\hline
Ctrl+Y & Вернуть действие \\
\hline
Ctrl+A & Выделить все \\
\hline
Ctrl+F & Найти по идентификатору \\
\hline
\multicolumn{2}{|c|}{\textbf{Панель SCg-инструментов}} \\
\hline
1 & Режим Выделение\\
\hline
2 & Режим Создание SCg-пары \\
\hline
3 & Режим Создание SCg-шины \\
\hline
4 & Режим Создание SCg-контура \\
\hline
5 & Выравнивание по сетке \\
\hline
6 & Выравнивание связки с шиной \\
\hline
7 & Выравнивание по вертикали \\
\hline
8 & Выравнивание по горизонтали \\
\hline
+ & Увеличить масштаб \\
\hline
- & Уменьшить масштаб \\
\hline
\multicolumn{2}{|c|}{\textbf{Графическая сцена}} \\
\hline
Ctrl+колесо мыши & Увеличение/уменьшение масштаба графической сцены \\
\hline
Ctrl+стрелка & Перемещение объекта \\
\hline
I & Установка идентификатора объекта \\
\hline
C & Установка содержимого узла \\
\hline
H & Отображение/скрытие содержимого узла \\
\hline
D & Удаление содержимого узла \\
\hline
Delete & Удаление элемента \\
\hline
Backspace & Удаление контура без удаления содержащихся в нем объектов \\
\hline
\end{tabular}
