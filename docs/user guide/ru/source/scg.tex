
\subsection{Редактор sc.g-текстов}

Описываемый в данном разделе редактор sc.g-текстов, работает с sc.g-текстами записанными с помощью SCg-кода версии 0.1.0.
Основная идея, которая преследуется в данном редакторе sc.g-текстов - это упрощение и ускорение процесса редактирования sc.g-текстов.
 
В процессе редактирования пользователю доступны различные режимы редактирования. Одинаковые пользовательские действия в различных
режимах редактирования могут приводить к различным результатам. Всего выделено 4 режима:

\begin{enumerate}
 \item \textbf{Режим выделения и создания узлов.}
 
 В данном режиме пользователь может работать со всеми объектами выделяя их, перемещая их, вызывая контекстное меню с командами. 
 Отличительной особенностью данного режима является то, что в нем можно создавать {\sf sc.g-узлы}. 
 Для этого необходимо выполнить двойной клик мышью в месте, где необходимо создать узел 
 (в момент создания под указателем мыши не должно быть других объектов);
 \item \textbf{Режим создания sc.g-пар (дуг).} 
 
 Создание sc.g-пары (дуги) начинается с того, что пользователь указывает объект из которого она будет выходить (клик левой клавиши мыши), 
 далее он может указать точки излома дуги (клик левой клавиши мыши, в пустом месте документа), 
 завершается создание указанием конечного объекта (клик левой клавиши мыши). Стоит отметить, что при наведении указателя мыши
 на конечный объект, линия изображающая создаваемую sc.g-пару (дугу) меняется с прерывистой на сплошную.
 В процессе создания пользователь может отменять последнее действие (указание начального объекта, точки излома) путем клика правой клавиши мыши;
 \begin{figure}[H]
    \begin{center}
     	\includegraphics[width=0.6\textwidth]{../images/scg/scg-create-pair.png}
	\caption{Иллюстрация этапов создания sc.g-пары (дуги)}
	\label{scg_illustration_create_pair}
    \end{center}
 \end{figure}
 
 \item \textbf{Режим создания sc.g-шин.}
 
 SC.g-шины используются для увелечения контактной площади узла, поэтому они могут создаваться лишь для sc.g-узлов. 
 Создание шины начинается с указания sc.g-узла (клик левой клавиши мыши), далее как и при создании sc.g-пар (дуг) указываются точки излома. 
 Завершается процесс создание sc.g-шины путем клика на последней точке излома (при наведении на нее, рисуется {\bf\it сплошная линия}).
 Как и при создании дуг пользователь может отменять последнее действие нажатием правой клавиши мыши;
 \begin{figure}[H]
    \begin{center}
	\includegraphics[width=0.6\textwidth]{../images/scg/scg-create-bus.png}
	\caption{Иллюстрация этапов создания sc.g-шины}
	\label{scg_illustration_create_bus}
    \end{center}
 \end{figure}

 \item \textbf{Режим создания sc.g-контуров.}
 
 Создание sc.g-контура начинается с указаний первой его точки (клик левой клавиши мыши в пустой области документа). 
 Далее, как и в случае с sc.g-парами (дугами) и sc.g-шинами, указываются точки.
 Завершается процесс кликом левой клавиши мыши на начальной точке (при наведении на нее рисуется {\bf\it сплошная линия}).
 Стоит отметить, что все объекты, которые попадут внутрь созданного контура, будут добавлены в него автоматически.
 Как и при создании дуг и шин пользователь может отменять последнее действие нажатием правой клавиши мыши;
 \begin{figure}[H]
    \begin{center}
	\includegraphics[width=0.6\textwidth]{../images/scg/scg-create-contour.png}
	\caption{Иллюстрация этапов создания sc.g-контура}
	\label{scg_illustration_create_contour}
    \end{center}
 \end{figure}

\end{enumerate}

Кроме режимов редактирования на панели инструментов присутсвует целый ряд команд:
\begin{enumerate}
	\item Команда \textbf{выравнивания объектов по сетке}. Данная команда позволяет выровнять объекты имеющиеся в документе по сетке. Размеры сетки указываются при инициировании команды в диалоге настроек.
\begin{figure}[h]
	\centering\includegraphics[height=10.98, height=5.75cm]{../images/gridaligment.png}
	\caption{Пример использования команды выравнивания по сетке}
	\label{gridaligment}
\end{figure}
В диалоговом окне, которое появляется после инициирования команды, можно установить желаемые размеры сетки (см. Рисунок~\ref{gridaligment}).
	\item Команда \textbf{выравнивания связки с шиной}. Данная команда позволяет выровнять узел у которого имеется шина, с выходящими (выходящими) дугами. Как и в случае выравнивания по сетке, при инифиировании команды появляется окно настроек (см. Рисунок~\ref{tuplealigment}). Для инициирования команды необходимо выделить узел из которого выходит шина.
	
\begin{figure}[h]
	\includegraphics[width=13.07cm, height=9.82cm]{../images/tuplealigment.png}
	\caption{Пример использования команды выравнивания связки с шиной}
	\label{tuplealigment}
\end{figure}
	\item Команда \textbf{выравнивания по вертикальной линии}. Инициирование команды приводит к выравниванию по X координате выделенных объектов. Новая координата считается как среднее арифметическое X координат выделенных объектов. Y координаты объектов не меняются. Команда не имеет настроек;
	\item Команда \textbf{выравнивания по горизонтальной линии}. Инициирование команды приводит к вырваниванию по Y координате выделенных объектов. Новая координата считается как среднее ариaметическое Y координат выделенных объектов. X координаты объектов не меняются. Команда не имеет настроек;
	\item Команда \textbf{увеличения масштаба}. Инициирование команды приводит к увеличению масштаба изображения;
	\item Список \textbf{стандартных масштабов}. Элемент управления, который позволяет выбрать масштаб из уже заранее имеющихся, либо указать свой масштаб;
	\item Команда \textbf{уменьшения масштаба}. Инициирование команды приводит к уменьшению масштаба изображения.
\end{enumerate}

\hrule
\smallskip
\noindent\includegraphics[width=25pt, height=25pt]{../images/note.png} \textcolor[rgb]{.67,.05,.05}{Каждый режим можно инициировать нажатием клавиш 1-4. Также и команды выравнивания клавишами 5-8.}
\smallskip
\hrule
\smallskip
\hrule
\smallskip
\noindent\includegraphics[width=25pt, height=25pt]{../images/note.png} \textcolor[rgb]{.67,.05,.05}{Рекомендуем использовать выравнивание по сетке и связок с шиной. Настроив единажды размеры сетки или параметры для шины, вы можете используя горячие клавиши 5 и 6 быстро ровнять конструкции.}
\smallskip
\hrule
\medskip
Кроме перечиленных выше команд существует еще целый ряд команд редактирования:
\begin{itemize}
	\item Команда \textbf{изменения основного текстового идентификатора элемента}. Любому sc.g-элементу можно устанавливать некоторый текстовый идентификатор. Для того, чтобы установить (изменить) текстовый идентификатор sc.g-элемента, необходимо в контекстном меню данного элемента выбрать пункт “\textbf{Изменить идентификатор}”, либо воспользоваться \textit{горячей клавишей} \textbf{I}. После чего будет вызвано диалоговое окно, в котором вы сможете ввести необходимый идентификатор (см. Рисунок~\ref{idtfdialog}).
	\begin{figure}[h]
		\centering\includegraphics[width=6.70cm, height=3.92cm]{../images/idtfdialog.png}
		\caption{Диалоговое окно изменения текстового идентификатора}
		\label{idtfdialog}
	\end{figure}
	\item Команда \textbf{изменения типа элемента}. Позволяет изменять тип sc.g-элемента (константность, структурный тип и т.д.). Доступные типы можно открыть нажатием правой клавиши мыши на элементе (тип которого будет изменяться) и выборе нужного типа в меню \textbf{Изменить тип};
	\item Команда \textbf{установки содержимого}. Установить содержимое для sc.g-узла достаточно просто, для этого необходимо  нажать правой клавишей мыши на узле и выбрать команду установки содержимого. После инициирования команды появляется диалог, в котором можно выбрать тип содержимого и само содержимое (см. Рисунок~\ref{contentdialog})
	\begin{figure}[h]
		\centering\includegraphics[width=6.09cm, height=7.46cm]{../images/contentdialog.png}
		\caption{Диалог установки содержимого}
		\label{contentdialog}
	\end{figure}
	\hrule
\smallskip
\noindent\includegraphics[width=25pt, height=25pt]{../images/lamp.png} \textcolor[rgb]{.25, .67, .2}{\textbf{Примечание}: Есть альтернативный и более простой способ установки содержимого из файла. Для этого надо лишь перетянуть файл на окно, после чего создастся sc.g-узел в содержимом которого будет установлен перетаскиваемый файл.}
\smallskip
\hrule
\end{itemize}
